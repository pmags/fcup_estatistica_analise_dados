% Options for packages loaded elsewhere
\PassOptionsToPackage{unicode}{hyperref}
\PassOptionsToPackage{hyphens}{url}
%
\documentclass[
]{article}
\usepackage{lmodern}
\usepackage{amssymb,amsmath}
\usepackage{ifxetex,ifluatex}
\ifnum 0\ifxetex 1\fi\ifluatex 1\fi=0 % if pdftex
  \usepackage[T1]{fontenc}
  \usepackage[utf8]{inputenc}
  \usepackage{textcomp} % provide euro and other symbols
\else % if luatex or xetex
  \usepackage{unicode-math}
  \defaultfontfeatures{Scale=MatchLowercase}
  \defaultfontfeatures[\rmfamily]{Ligatures=TeX,Scale=1}
\fi
% Use upquote if available, for straight quotes in verbatim environments
\IfFileExists{upquote.sty}{\usepackage{upquote}}{}
\IfFileExists{microtype.sty}{% use microtype if available
  \usepackage[]{microtype}
  \UseMicrotypeSet[protrusion]{basicmath} % disable protrusion for tt fonts
}{}
\makeatletter
\@ifundefined{KOMAClassName}{% if non-KOMA class
  \IfFileExists{parskip.sty}{%
    \usepackage{parskip}
  }{% else
    \setlength{\parindent}{0pt}
    \setlength{\parskip}{6pt plus 2pt minus 1pt}}
}{% if KOMA class
  \KOMAoptions{parskip=half}}
\makeatother
\usepackage{xcolor}
\IfFileExists{xurl.sty}{\usepackage{xurl}}{} % add URL line breaks if available
\IfFileExists{bookmark.sty}{\usepackage{bookmark}}{\usepackage{hyperref}}
\hypersetup{
  pdftitle={Titel},
  pdfauthor={name},
  hidelinks,
  pdfcreator={LaTeX via pandoc}}
\urlstyle{same} % disable monospaced font for URLs
\usepackage[margin=2.5cm]{geometry}
\usepackage{longtable,booktabs}
% Correct order of tables after \paragraph or \subparagraph
\usepackage{etoolbox}
\makeatletter
\patchcmd\longtable{\par}{\if@noskipsec\mbox{}\fi\par}{}{}
\makeatother
% Allow footnotes in longtable head/foot
\IfFileExists{footnotehyper.sty}{\usepackage{footnotehyper}}{\usepackage{footnote}}
\makesavenoteenv{longtable}
\setlength{\emergencystretch}{3em} % prevent overfull lines
\providecommand{\tightlist}{%
  \setlength{\itemsep}{0pt}\setlength{\parskip}{0pt}}
\setcounter{secnumdepth}{5}
%----------------------------------------------------------------------------------------
%	REQUIRED PACKAGES
%
% pdfpages:(http://mirrors.up.pt/pub/CTAN/macros/latex/contrib/pdfpages/pdfpages.pdf)
% geometry: (https://www.overleaf.com/learn/latex/Page_size_and_margins)
% fancyhdr: (https://www.overleaf.com/learn/latex/Headers_and_footers)
% xcolor: (https://www.overleaf.com/learn/latex/Using_colours_in_LaTeX)
% titlesec: https://www.overleaf.com/learn/latex/Sections_and_chapters
%----------------------------------------------------------------------------------------

\usepackage{pdfpages} %allows the introduction of pdf pages 
\usepackage{fancyhdr} %changes headers and footers
\usepackage{xcolor} %Lets set different colors for the document
\usepackage{titlesec} %customizes chapters(book) and sections(article)
\usepackage{tikz}
\usepackage{fontspec} %customizes fonts
\usepackage{graphicx} %add graphical elements
\usepackage{background} %customizes documents backgrounds
\usepackage{lastpage}
\usepackage{caption} % allows image caption costumization
\usepackage{blindtext}
\usepackage[most]{tcolorbox}
\tcbuselibrary{skins}
\usepackage{multicol} % allows more than one column on text
\usepackage{enumitem} % Allows to change lists %>% 
\usepackage[default]{opensans}
\usepackage{setspace}\setstretch{1.25}
\usepackage{float}
\floatplacement{figure}{H}
%\usepackage{setspace}\singlespacing

%----------------------------------------------------------------------------------------
%	FONTS FORMAT
%----------------------------------------------------------------------------------------

\setmainfont[SizeFeatures={Size=10}]{Open Sans}

%----------------------------------------------------------------------------------------
%	DOCUMENT COLORS
%----------------------------------------------------------------------------------------

\definecolor{DarkPurple}{HTML}{680c62}
\definecolor{MediumPurple}{HTML}{a8139d}
\definecolor{LightPurple}{HTML}{c26cf0}
\definecolor{Grey}{HTML}{bfbfbf}
\definecolor{lightGreen}{HTML}{ddeedd}
\definecolor{mediumGreen}{HTML}{77bb77}

\definecolor{DarkGray}{HTML}{403D3C}
\definecolor{AtomicGreen}{HTML}{10FF10}

%----------------------------------------------------------------------------------------
%	SECTION FORMAT
%----------------------------------------------------------------------------------------

\titleformat{\section}[block]
  {\titlerule\addvspace{4pt}\normalfont\fontsize{12}{12}\bfseries}
  {\thesection\enspace}{0pt}{}[\vspace{2pt}\titlerule]

\titleformat*{\section}{\color{DarkGray}\fontsize{16}{16}\normalfont\bfseries}
\titleformat*{\subsection}{\color{DarkGray}\fontsize{12}{12}\normalfont\bfseries}
\titleformat*{\subsubsection}{\color{AtomicGreen}\fontsize{10}{10}\normalfont\bfseries}

\titlespacing*{\section}{0pt}{30pt}{50pt}
\titlespacing*{\subsection}{0pt}{40pt}{10pt}

%----------------------------------------------------------------------------------------
%	CUSTOM HEADER AND FOOTER
%----------------------------------------------------------------------------------------

\pagestyle{fancy}
\fancyhf{} %clear all headers and footer fields
%\fancyhead[RO,RE]{\textcolor{DarkGray}{math.marketing}}
\fancyhead[LO,LE]{\textcolor{DarkGray}{\thepage}}
\renewcommand{\headrulewidth}{0pt} %Eliminates header row that is part of the default


\backgroundsetup{
  scale=1.3,
  opacity=1,
  angle=0,
  position=current page.south,
  hshift=0pt, %-27
  vshift=20pt,
  contents={%
  \small\sffamily%
  \begin{minipage}{1\textwidth}
  \includegraphics[scale=0.5]{theme/images/footer.png}
  \end{minipage}%
  }
}

%----------------------------------------------------------------------------------------
%	CUSTOM IMAGE FORMATING
%----------------------------------------------------------------------------------------

\captionsetup[figure]{name=Fig.}

%----------------------------------------------------------------------------------------
%	COVER PAGE
%----------------------------------------------------------------------------------------

\AtBeginDocument{\let\maketitle\relax}

%----------------------------------------------------------------------------------------
%	NEW COMMANDS
%-------------------------------------------------------------------------------------

% http://texdoc.net/texmf-dist/doc/latex/tcolorbox/tcolorbox.pdf

\newcommand{\impactBox}[2]{
  \begin{tcolorbox}[width=\textwidth,
  fonttitle = \sffamily\bfseries\large, 
  colframe={mediumGreen},
  colback={lightGreen},
  title={#1},
  sharp corners]  
  \textsc{#2}
  \end{tcolorbox} 
}

\newcommand{\attentionBox}[1]{
  \begin{tcolorbox}[fonttitle = \sffamily\bfseries\large,
  bicolor,
  sidebyside,
  lefthand width=0.5cm,
  sharp corners,
  boxrule=.4pt,
  colback=yellow!30,
  colbacklower=yellow!20]
    \includegraphics[scale=5]{theme/images/attention-sign.png}
  \tcblower
    \textsc{#1}%
  \end{tcolorbox}
}
\usepackage[]{natbib}
\bibliographystyle{plainnat}

\title{Titel}
\usepackage{etoolbox}
\makeatletter
\providecommand{\subtitle}[1]{% add subtitle to \maketitle
  \apptocmd{\@title}{\par {\large #1 \par}}{}{}
}
\makeatother
\subtitle{Subtitle}
\author{name}
\date{date}

\begin{document}
\NoBgThispage %does not apply background
\includepdf{theme/images/cover.pdf}
\maketitle



{
\setcounter{tocdepth}{3}
\tableofcontents
}
\hypertarget{introduction}{%
\section{Introduction}\label{introduction}}

\hypertarget{goals-and-objectives}{%
\subsection{Goals and objectives}\label{goals-and-objectives}}

Text

\hypertarget{assumptions}{%
\subsection{Assumptions}\label{assumptions}}

Text

\hypertarget{definitions}{%
\subsection{Definitions}\label{definitions}}

Text

\newpage

\newpage

\hypertarget{data-exploration}{%
\section{Data Exploration}\label{data-exploration}}

\hypertarget{holetype}{%
\subsection{HoleType}\label{holetype}}

\begin{center}\includegraphics{_main_files/figure-latex/unnamed-chunk-4-1} \end{center}

A variável \textbf{Holetype} que representa o tipo do buraco feito para coleta da amostra de cromitita em uma das sessõe da mina do complexo de Bushveld. A variável é categórica sendo que a maior prediminância da classe sendo o buraco feito por meio de um \textbf{poço artesiano} e a outra classe sendo o buraco feito por meio de \textbf{deflexão}.

\hypertarget{maxdepth}{%
\subsection{MaxDepth}\label{maxdepth}}

\begin{center}\includegraphics{_main_files/figure-latex/unnamed-chunk-5-1} \end{center}

\textbf{MaxDepth} é a profundidade máxima do buraco utilizado para coleta da amostra de cromitita para a análise química. A váriável é contínua com distribuição multimodal, vemos também que o box-plot não apresenta presença de outliers e aparentemente a média e o desvio padrão são boas medidas de tendência central.

\hypertarget{depthfrom}{%
\subsection{DepthFrom}\label{depthfrom}}

\begin{center}\includegraphics{_main_files/figure-latex/unnamed-chunk-6-1} \end{center}

\textbf{DepthFrom} é a profundidade inicial de onde a amostra de cromitita foi coletada para a análise química. A váriável é contínua com distribuição gamma. O box-plot indica a presença de outliers, mas no contexto da coleta de amostras de cromitita para análise quimica esses valores não se confirmam como outliers e sim como prováveis de acontecer. Observamos por meio da distribuição desta variável que temos alguns valores com grande magnitude e desta forma a mediana e os valores máximo e mínimo são boas medidas de tendência central.

\hypertarget{depthto}{%
\subsection{DepthTo}\label{depthto}}

\begin{center}\includegraphics{_main_files/figure-latex/unnamed-chunk-7-1} \end{center}

\textbf{DepthTo} é a profundidade final de onde a amostra de cromitita foi coletada para a análise química. A váriável é contínua com distribuição gamma. O box-plot indica a presença de outliers, mas no contexto da coleta de amostras de cromitita para análise quimica esses valores não se confirmam como outliers e sim como prováveis de acontecer. Observamos por meio da distribuição desta variável que temos alguns valores com grande magnitude e desta forma a mediana e os valores máximo e mínimo são boas medidas de tendência central.

\hypertarget{cr2o3_}\label{cr2o3_}}

\begin{center}\includegraphics{_main_files/figure-latex/unnamed-chunk-8-1} \end{center}

\textbf{Cr2O3\_\%} representa o percentual de \textbf{óxido de cromo} contido na amostra de cromitita analisada. A váriável é contínua com distribuição bimodal, o box-plot sugere a presença de outliers. Por possuir alguns valores de grande amplitude a mediana e os valores máximo e mínimo são boas medidas de tendência central.

\hypertarget{feo_}\label{feo_}}

\begin{center}\includegraphics{_main_files/figure-latex/unnamed-chunk-9-1} \end{center}

\textbf{FeO\_\%} representa o percentual de \textbf{óxido de ferro} contido na amostra de cromitita analisada. A váriável é contínua com distribuição normal, o box-plot sugere a presença de outliers. Como a distribuição desta variável se assemelha a uma distribuição normal a média e o desvio padrão são boas medidas de tendência central.

\hypertarget{sio2_}\label{sio2_}}

\begin{center}\includegraphics{_main_files/figure-latex/unnamed-chunk-10-1} \end{center}

\textbf{SiO2\_\%} representa o percentual de \textbf{dióxido de silício} contido na amostra de cromitita analisada. A váriável é contínua com distribuição bimodal, o box-plot sugere a presença de outliers. E, por possuir alguns valores de grande amplitude a mediana e os valores máximo e mínimo são boas medidas de tendência central.

\hypertarget{mgo_}\label{mgo_}}

\begin{center}\includegraphics{_main_files/figure-latex/unnamed-chunk-11-1} \end{center}

\textbf{MgO\_\%} representa o percentual de \textbf{óxido de magnésio} contido na amostra de cromitita analisada. A váriável é contínua com distribuição normal, o box-plot sugere a presença de outliers. Como a distribuição desta variável se assemelha a uma distribuição normal a média e o desvio padrão são boas medidas de tendência central.

\hypertarget{al2o3_}\label{al2o3_}}

\begin{center}\includegraphics{_main_files/figure-latex/unnamed-chunk-12-1} \end{center}

\textbf{Al2O3\_\%} representa o percentual de \textbf{óxido de alumínio} contido na amostra de cromitita analisada. A váriável é contínua com distribuição bimodal, o box-plot sugere a presença de outliers. E, por possuir alguns valores de grande amplitude a mediana e os valores máximo e mínimo são boas medidas de tendência central.

\hypertarget{cao_}\label{cao_}}

\begin{center}\includegraphics{_main_files/figure-latex/unnamed-chunk-13-1} \end{center}

\textbf{CaO\_\%} representa o percentual de \textbf{óxido de cálcio} contido na amostra de cromitita analisada. A váriável é contínua com distribuição bimodal, o box-plot sugere a presença de outliers. E, por possuir alguns valores de grande amplitude a mediana e os valores máximo e mínimo são boas medidas de tendência central.

\hypertarget{p_}\label{p_}}

\begin{center}\includegraphics{_main_files/figure-latex/unnamed-chunk-14-1} \end{center}

\textbf{P\_\%} representa o percentual de \textbf{fósfoto} contido na amostra de cromitita analisada. A váriável é contínua com distribuição similar a uma gamma ou poisson, o box-plot sugere a presença de outliers. E, por possuir alguns valores de grande amplitude a mediana e os valores máximo e mínimo são boas medidas de tendência central.

\hypertarget{au_icp_ppm}{%
\subsection{Au\_ICP\_ppm}\label{au_icp_ppm}}

\begin{center}\includegraphics{_main_files/figure-latex/unnamed-chunk-15-1} \end{center}

\textbf{Au\_ICP\_ppm} representa as partes por milhão de ICP (Plasma Acoplado Indutivamente) de \textbf{ouro} contido na amostra de cromitita analisada. A váriável é contínua com distribuição similar a uma gamma ou poisson, o box-plot sugere a presença de outliers. E, por possuir alguns valores de grande amplitude a mediana e os valores máximo e mínimo são boas medidas de tendência central.

\hypertarget{pt_icp_ppm}{%
\subsection{Pt\_ICP\_ppm}\label{pt_icp_ppm}}

\begin{center}\includegraphics{_main_files/figure-latex/unnamed-chunk-16-1} \end{center}

\textbf{Pt\_ICP\_ppm} representa as partes por milhão de ICP (Plasma Acoplado Indutivamente) de \textbf{platina} contido na amostra de cromitita analisada. A váriável é contínua com distribuição bimodal, o box-plot sugere a presença de outliers. E, por possuir alguns valores de grande amplitude a mediana e os valores máximo e mínimo são boas medidas de tendência central.

\hypertarget{pd_icp_ppm}{%
\subsection{Pd\_ICP\_ppm}\label{pd_icp_ppm}}

\begin{center}\includegraphics{_main_files/figure-latex/unnamed-chunk-17-1} \end{center}

\textbf{Pd\_ICP\_ppm} representa as partes por milhão de ICP (Plasma Acoplado Indutivamente) de \textbf{paládio} contido na amostra de cromitita analisada. A váriável é contínua com distribuição gamma, o box-plot sugere a presença de outliers. E, por possuir alguns valores de grande amplitude a mediana e os valores máximo e mínimo são boas medidas de tendência central.

\hypertarget{rh_icp_ppm}{%
\subsection{Rh\_ICP\_ppm}\label{rh_icp_ppm}}

\begin{center}\includegraphics{_main_files/figure-latex/unnamed-chunk-18-1} \end{center}

\textbf{Rh\_ICP\_ppm} representa as partes por milhão de ICP (Plasma Acoplado Indutivamente) de \textbf{ródio} contido na amostra de cromitita analisada. A váriável é contínua com distribuição gamma, o box-plot sugere a presença de outliers. E, por possuir alguns valores de grande amplitude a mediana e os valores máximo e mínimo são boas medidas de tendência central.

\hypertarget{ir_icp_ppm}{%
\subsection{Ir\_ICP\_ppm}\label{ir_icp_ppm}}

\begin{center}\includegraphics{_main_files/figure-latex/unnamed-chunk-19-1} \end{center}

\textbf{Ir\_ICP\_ppm} representa as partes por milhão de ICP (Plasma Acoplado Indutivamente) de \textbf{irídio} contido na amostra de cromitita analisada. A váriável é contínua com distribuição bimodal, o box-plot sugere a presença de outliers. E, por possuir alguns valores de grande amplitude a mediana e os valores máximo e mínimo são boas medidas de tendência central.

\hypertarget{ru_icp_ppm}{%
\subsection{Ru\_ICP\_ppm}\label{ru_icp_ppm}}

\begin{center}\includegraphics{_main_files/figure-latex/unnamed-chunk-20-1} \end{center}

\textbf{Ru\_ICP\_ppm} representa as partes por milhão de ICP (Plasma Acoplado Indutivamente) de \textbf{ruténio} contido na amostra de cromitita analisada. A váriável é contínua com distribuição gamma, o box-plot sugere a presença de outliers. E, por possuir alguns valores de grande amplitude a mediana e os valores máximo e mínimo são boas medidas de tendência central.

\textbf{Observação:} para todas as variáveis que representam as análises químicas tanto o percentual quanto as partes por milhão de ICP (Plasma Acoplado Indutivamente) vimos que o box-plot sugere a presença de outliers, porem esses valores atípicos podem desafiar um esquema de discriminação, mas parecem ser bastante comuns no Complexo Bushveld e são comumente atribuídos a heterogeneidades locais nos cromititos, por exemplo, a presença de grandes oikocristais de piroxênio ou uma concentração elevada de plagioclásio. E, por se tratar de fenômenos que podem ocorrer, nós não iremos eliminar os outlies dessas variáveis.

\hypertarget{stratigraphy}{%
\subsection{Stratigraphy}\label{stratigraphy}}

\begin{center}\includegraphics{_main_files/figure-latex/unnamed-chunk-21-1} \end{center}

\textbf{Stratigraphy} representa a posição estratigráfica das camadas de cromitita dentro da \textbf{Zona Crítica} a partir da base para cima. A váriável é categorica multiclasse. Através do gráfico de frequência observamos que existem 6 zonas críticas pouco frequentes, 4 zonas críticas com frequência média e 4 zonas críticas de alta frequência.

\hypertarget{martriz-de-correlauxe7uxe3o}{%
\subsection{Martriz de Correlação}\label{martriz-de-correlauxe7uxe3o}}

\begin{center}\includegraphics{_main_files/figure-latex/unnamed-chunk-22-1} \end{center}

\textbf{Não esquecer limpar na}

\newpage

\end{document}
